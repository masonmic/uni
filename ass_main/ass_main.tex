\documentclass[11pt,a4paper]{report}
\usepackage[margin=0.7in]{geometry}


\begin{document}

\title{Report of Industrial Year at HP\\Department of Computer Science\\ University of Aberystwyth\\ \texttt{mim20@aber.ac.uk}} \date{\today}
\maketitle

\begin{abstract} 

Working for Hewlett Packard (will now be refereed to as HP) For 14 months. During time spent worked with two main service teams. Cinder and bock. Cinder, the openstack and opensourced client that provides API for block storage solutions to be consumed by NOVA, the openstack, opensource compute host for cloud solutions, and Bock, the block storage solution implemented within the HP public cloud. Also worked on Helion, private cloud solutions within Cinder service team\\ 
Producing code for bug fixing, features, testing, automation for upstream (openstack(cinder)) downstream(bock)and upstream internal(Helion)while also doing peer code review for mentioned services. 
Working on 24/7 on call support for the public cloud to ensure that services do not experience outage and working closely with technical operations (will now be refereed to as tech ops), to ensure services experience no outage is something I worked towards and successfully integrated myself into.

\end{abstract}

\tableofcontents
\newpage
\section{Organisational Environment}
 


\section{Technical and Application Environments}

% HARDWARE; Workstation (z400, z640), public cloud instances, public cloud hardware, galway lab, laptop, connections(VPN's, wifi)
% SOFTWARE; Communications-(ubuntu, windows (office;outlook, lync..), hipchat, mobile) python virtual enviroment, HLM, vim, Jira, trello. devstack
% WORKFLOW; Gerrit, git. HLM, pulic cloud, upstream openstack) 

The cinder service was consumed by multiple projects; Helion, HP public cloud and the openstack project, Cinder. Cinder was used as the API service in both Helion and HP public cloud, whereas openstack cinder is an open source project that provides the code base for cinder. Three areas will be discussed to collaborate the technical and the application of tools and services used. They will be;

\begin{itemize}
\item Hardware
\item software
\item work-flow
\end{itemize} 

\subsection{Hardware}
The hardware that was used is very diverse. From local systems; laptop and desktop to SSH sessions to production systems in Vegas, there was a broad spectrum.
Base development computer was a HP z640. This included duel Intel Xeon 3.50GHz processor, 128Gb (8 x 16gb) DDR4 2133MHz RAM, 256 Solid state drive (OS installed), 2TB hard drive, 1Gbit/s network interface. The operating system was ubuntu 14.4 LTS whice was dispayed on dueal 32'ch IPS monitors. The machine was an upgrade from a z400 which saw a huge increase in RAM. The z400 also did not support virtuialaztion. This workstation was used specifically for its large volume of RAM to help create virtual environments for the private cloud deployment system; Helion life-cycle manager (will now be referred to as HLM).

\subsection{Software}

\subsection{Work-flow}


\section{What I did}



\section{Evaluation}



\section{•}

\end{document}